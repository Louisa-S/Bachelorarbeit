
\documentclass[12pt]{article}
\begin{document}

\tableofcontents

\section{Introduction}

miRNAs are small, non-coding RNAs that function in post-transcriptional regulation of gene expression, especially in terms of gene silencing. A miRNA consists of approximately 22 nucleotides, sometimes up to 28, which build a single strand RNA. It is mainly active in combination with a RNA-induced silencing complex (RISC). This complex targets a mRNA at its 3' UTR by complementary binding to the sequence. This results in repression of mRNA translation or degradation of the respective mRNAs. \\
This regulation plays an important role for detecting diseases.
(Because we want to understand their biological function and regulation mechanism we have to elucidate their respective target genes.)
The main challenge is finding and predicting new targets of miRNAs. This important problem is still not solved reliably.\\
One of the first characteristics that were discovered was the occurrence of a "seed region" close to the 5' end of the miRNA. This region consists of 7 or 8 nucleotides which are complementary to the respective nucleotides in the mRNA. Another characteristic is the bulge after the complementary region where the nucleotides don't fit together. After this part complentary bases towards the 3' end of the miRNA may occur again. \\
Whether these rule apply and are strong evidence for a prediction this paper will show. \\
A collection of experimental validated miRNA-tagrget interactions(MTIs) is essential. The database miRTarBase provides about 7500 strong validated MTIs and 348 000 weak ones from different species. Different experiment types were used to validate the data, including Reporter assay, Western plot, qPCR, Microarray, NGS, pSILAC and other Methods where the first three are the one that deliver strong evidences.

\\

mirtarbase: \\
http://nar.oxfordjournals.org/content/early/2015/11/19/nar.gkv1258.full \\ 

http://www.ncbi.nlm.nih.gov/pmc/articles/PMC3013699/ \\



\\
[1] Research; MicroRNA targets in Drosophila;
Anton J Enright, Bino John, Ulrike Gaul, Thomas Tuschl, Chris Sander and Debora S Marks 
\end{document}